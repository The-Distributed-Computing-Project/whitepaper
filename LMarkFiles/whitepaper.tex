\documentclass[12pt,english]{article}
\usepackage{hyperref}
\usepackage{graphicx}
\usepackage{indentfirst}
\usepackage{color}
\usepackage{listings}

\graphicspath{ {~/Pictures} {./Images}}

\usepackage[a4paper,bindingoffset=0.2in,left=1in,right=1in,top=1in,bottom=1in,footskip=.25in]{geometry}

\begin{document}


\begin{title}

\centering

\section*{\LARGE{AetherGrid}}
\normalsize{} 
\phantomsection 
\addtocontents{toc}{\protect\setcounter{tocdepth}{2}} 
\addcontentsline{toc}{section}{AetherGrid} 

\section*{\large{Samuel Hastings}}
\normalsize{} 
\phantomsection 
\addtocontents{toc}{\protect\setcounter{tocdepth}{4}} 
\addcontentsline{toc}{subsubsection}{Samuel Hastings} 

\section*{\large{January 24, 2025}}
\normalsize{} 
\phantomsection 
\addtocontents{toc}{\protect\setcounter{tocdepth}{4}} 
\addcontentsline{toc}{subsubsection}{January 24, 2025} 



\end{title}

\noindent \vspace{5mm} %5mm vertical space

\section*{\huge{1 Introduction}}
\normalsize{} 



Most "crypto-currencies" currently in widespread use have far too many drawbacks, and too little real-world utility. The public's view of them is typically as: a method of investing, a volatile and risky place to put money, a method utilized for "get-rich-quick" schemes, or other, typically negative connotations.


The original goals, set by originals in the market such as Bitcoin, are put simply, for a decentralized currency, completely reliant on the combined efforts of the network, rather than a single entity. Bitcoin's methodology for implementation is inherently flawed though. The price to purchase is not only extremely volatile, but generally trends upward at a far greater rate than inflation. It is also not easily expandable, and impossible to last for a long duration. It is also arguable that despite existing so long, it is still far from attaining widespread adoption outside of dedicated investment applications. Another major issue is the algorithm which verifies the network. The most common method for this algorithm was Proof-of-Work, which requires an enormous amount of computing power, and therefore energy. Another popular method is called Proof-of-Stake, which requires staking an amount of the currency as collateral.


The proposed system will not only attempt to fix many of the stated flaws regarding existing cryptocurrencies, it will also describe the basis of a new type of Proof-of-Work, which solves issues with current algorithms and benefits the wider population.

\noindent \vspace{5mm} %5mm vertical space
\section*{\huge{2 Proof of Tangential Work}}
\normalsize{} 



Typical proof-of-work algorithms utilize the verification of the blockchain itself as the work to be done. Realistically, while this is a neat and fitting piece of work, as it is inherent to the system, it is also inherently wasteful. AetherGrid would rather have the work being done a type of general computation that serves a use.
\end{document}